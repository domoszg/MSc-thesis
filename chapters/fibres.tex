\documentclass{standalone}
\usepackage{diplomski}

\begin{document}

\chapter{Fibre-optic technology} \label{ch:fibres}
\pagenumbering{arabic}
\setcounter{page}\thestranica

% --------------------------------------

Optical fibre enables propagation of light by means of total reflection occurring within its core. Fibres consist of a number of cylindrical layers, the innermost being the core, surrounded by cladding. The two layers are further enclosed in a jacket, following a number of shields and isolation materials, depending on the required cable rigidity and specified applications. A number of standard fibre-optic cables are used in communication systems.

single mode

multimode

\section{Fibre losses and linear effects}

In modern communication applications, optical fibres impose as the best data transfer medium. The qualities of an optical fibre supersede those of electrical low- and high-frequency conductors. Firstly, optical fibres employ a much higher frequency range, enabling broad-spectra data transfer. The medium is, thus, utilized to a much greater extent than, for example, a coaxial cable operating in ultra-high frequency range under 1 GHz. Also, optical fibres have proven to be much more durable in terms of both mechanical endurance and immunity to electromagnetic interference. In communication and sensing applications, optical fibres represent a reliable, resilient medium. \\

In normal conditions, optical power in the fibre attenuates as governed by Beer's law. Optical power at any point in the fibre can be expressed by
\begin{equation}
P(z_0) = P_\textrm{in} \exp\left(-\alpha z_0\right) \textrm{,}
\end{equation}
where $\alpha$ is the attenuation coefficient in units of Np/km. For convenience, $\alpha$ is usually expressed in units of dB/km. Losses in a fibre depend on the optical wavelength. In communication systems, three \emph{optical communication windows} were defined as wavelengths where silica fibre exhibits minimum losses. The first around 850 nm (not as used in modern systems), the second around 1310 nm and the third around 1550 nm. These windows arise as a result of wavelength dependence of fibre material absorption, which represents the absorption by fused silica and due to the presence of impurities. The other fundamental loss mechanism is Rayleigh scattering. It arises from microscopic density fluctuations of silica that lead to refractive index fluctuations on a scale smaller than the optical wavelength. The contribution of Rayleigh scattering to the overall fibre loss can be expressed as
\begin{equation}
\alpha_\textrm{Ray} = C / \lambda^4 \textrm{.}
\end{equation}
Rayleigh scattering is, thus, largely dependent on wavelength. \\


\section{Non-linear optical effects}

\subsection{Kerr effect}
The three most important non-linear effects in optical fibres are described here.

\subsection{Brillouin scattering}

In distributed temperature sensing (abbr. DTS) systems, either of the two non-linear effects can be effectively used. The first is the Brillouin effect. This effect is a result of the electrostriction phenomenon, i.e. the compression of the material in the presence of an electromagnetic field. The optical pulse functions as a pump, exciting molecules in the fibre from the ground state into a higher, virtual state. The decay of these states will produce a Stokes component -- with the wavelength being longer that that of the incident optical wave -- or an anti-Stokes component, with the wavelength being shorter than that of the incident optical wave. An illustration of this shift is given in Figure \ref{fig:freq_shift}. 
%\label{fig:freq_shift}
This shift will create an acoustic wave at frequency
\begin{equation}
\Omega = \left|k_A\right| \, v_A = 2 v_A \, \left|k_p\right| \, \sin\left(\theta / 2 \right) \textrm{,}
\end{equation}
where wave vectors $k_A = k_P - k_S$ correspond to acoustic, pump and Stokes waves, $v_A$ the acoustic velocity, and the angle $\theta$ represents the angle between the pump and the scattered waves. Note that, due to the $\sin\left(\theta / 2\right)$ expression, the produced wave will not propagate in forward direction, but will instead be maximum in backward direction. The Brillouin shift is given as
\begin{equation}
\nu_B = \pm \frac{\Omega_B}{2 \pi} = \pm \frac{2 n_g v_A}{\lambda_P} \textrm{.}
\end{equation}
For silica fibre, and a typical pump wavelength of 1.55 \textmu m, the frequency is about 
\begin{equation}
\nu_B = \SI{11.1}{\giga \hertz} \textrm{.}
\end{equation}

The value of the frequency shift is temperature- and strain-dependent. Typically, the Brillouin frequency shift is tied to these physical constants at rates about
\begin{equation}
C_{\nu\theta} = \SI{1.10}{\mega \hertz \, \kelvin^{-1}} \textrm{,}
\end{equation}
\begin{equation}
C_{\nu\epsilon} = \SI{48}{\kilo \hertz \, \epsilon^{-1}} \textrm{.}
\end{equation}
The width of Stokes and anti-Stokes lines is typically 30--40 MHz. Monitoring the central frequencies, one can obtain the information about temperature and strain along the fibre. \\

The described method will be efficient in the spontaneous Brillouin scattering regime. In this regime, the scattered wave is generated spontaneously, as a result of fibre molecules being already thermally excited. Provided the incident, i.e. pump, optical power is large enough, the produced wave will beat with the pump wave, increasing the amplitude of the scattered wave. Such a phenomenon is then referred to as the stimulated Brillouin scattering (abbr. SBS). In order for the simulated scattering to take effect, the pump pulse power must be over the threshold level. This level is dependent on the Brillouin gain $g_B$, and the effective cross-section area of the fibre in question 
\begin{equation}
A_\textrm{eff} = \pi \, w^2 \textrm{,}
\end{equation}
with $w$ being the spot size. The threshold is given as
\begin{equation}
P_\textrm{th} \approx 21 \cdot \frac{A_\textrm{eff}}{g_B \, L_\textrm{eff}} \textrm{,}
\end{equation}
where $L_\textrm{eff}$ is the effective interaction length, found as
\begin{equation}
L_\textrm{eff} = \frac{1 - \exp\left(-\alpha \, L\right)}{\alpha} \textrm{.}
\end{equation}
Some methods for measuring temperature or strain by using stimulated Brillouin scattering , employing an excitation laser at each end of the measurement fibre, exist %\cite Rogers 1999
. This, however, requires excellent source coherence, and it is also impossible to measure temperature and strain simultaneously. Such a measurement fibre could be hardly used in communication purposes in parallel. The stimulated Brillouin scattering method, also dubbed \textit{loss Brillouin OTDR}, can nevertheless prove advantageous in long-haul measurements.


\subsection{Raman scattering}
% Druge frekvencije od bs
% rs ide u svim smjerovima, bs samo u back


% Mjerenje temperature
% Ramanova metoda
% Objasniti fenomen


% Spontaneous and stimulated


% Sirokopojasnost Ramana

% --------------------------------------

\setcounter{stranica}{\thepage}
\addtocounter{stranica}{1}

\end{document}