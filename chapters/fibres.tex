\documentclass{standalone}
\usepackage{diplomski}

\begin{document}

\chapter{Fibre-optic technology} \label{ch:fibres}
\pagenumbering{arabic}
\setcounter{page}\thestranica

% --------------------------------------

Optical fibre enables propagation of light by means of total reflection occurring within its core. Fibres consist of a number of cylindrical layers, the innermost being the core, surrounded by cladding. The two layers are further enclosed in a jacket, following a number of shields and isolation materials, depending on the required cable rigidity and specified applications. A number of standard types of fibre-optic cables are used in communication systems. These types differ by their geometry and the type of optical signal they are able to guide. \\

In order for the light to be guided within the fibre, total reflection must be ensured, in order to keep the light within the fibre as it reaches the boundary of the core. A geometric-optical interpretation of this mechanism is presented in Figure \ref{fig:critical_angle}.
%\label{fig:critical_angle}
A critical angle $\phi_C$ is defined, as the maximum incident angle of the light to the core boundary, so that the light is still reflected against it. The angle is defined as
\begin{equation}
\sin \phi_C = n_2/n_1 \textrm{,}
\end{equation}
where $n_1$ is the refractive index of the core, and $n_2$ the refractive index of the cladding. Another characteristic of fibres is the \textit{numerical aperture} that represents the light gathering capacity of a fibre. It can be expressed as 
\begin{equation}
\textrm{NA} = n_1 \, \sqrt{2 \varDelta}
\end{equation}
\begin{equation}
\varDelta = \frac{n_1 - n_2}{n_1} \textrm{.}
\end{equation}
These parameters also define the maximum bitrate and length the fibre is able to transfer. As the light propagates, multiple geometrical paths can occur, leading to an arrival-time delay. This phenomenon is referred to as \textit{multipath dispersion}, and is a limiting factor of the bitrate-length multiple. To reduce the effects of multipath dispersion, graded-index fibres can be used, as opposed to step-index fibres. These are presented in Figure \ref{fig:graded_index}.
%\label{fig:graded_index}
%Pokazi step i graded
The gradual decrease in core's refractive index leads to better light confinement towards the centre axis of the fibre core. Thus, multipath dispersion effects will be smaller.\\

Apart from the geometrical interpretation, light propagation in fibres has to be reviewed as wave propagation, as well. It governed by the wave equation
\begin{equation} \label{eq:wave}
\nabla^2 \mathbf{E} + n^2(\omega) \, k_0^2 \, \mathbf{E} = 0 \textrm{.}
\end{equation}
The refractive index is taken as wavelength-dependant, and the free-space wave number $k_0$ is
\begin{equation}
k_0 = \frac{\omega}{c} = \frac{2 \pi}{\lambda} \textrm{.}
\end{equation}
By solving the equation \ref{eq:wave}, one can obtain multiple solutions, each corresponding to one \textit{fibre mode} that is uniquely determined by its propagation constant $\beta_{nm}$. Each mode has its own \textit{mode index}, or \textit{effective index}, that is determined as
\begin{equation}
\bar{n} = \beta/k_0 \textrm{,}
\end{equation}
and whose value lies within $n_1 > \bar{n} > n_2$. At $\bar{n} = n_2$, the mode in question is said to be in \textit{cut-off}, and is no longer guided. The parameter $V$, called the \textit{normalized frequency}, is used to determine cut-off condition.
\begin{equation}
V = k_0 \, a \, \sqrt{n_1^2 - n_2^2} \approx \frac{2 \pi}{\lambda} a n_1 \sqrt{2 \varDelta}
\end{equation}
Here, $a$ is the core radius. \\

For fibres with $V < 2.405$ only the fundamental mode is guided. Such fibres are, therefore, called \textit{single-mode} fibres. Taking typical values of $n_1 = 1.45$, $\varDelta = 5 \times 10^{-3}$, $\lambda = \SI{1.55}{\micro \meter}$, we find that the core radius should be no larger than $a < \SI{4.09}{\micro \meter}$. Fibres with a large value of $V$ are called \textit{multi-mode} fibres, as they guide a larger number of modes.


\section{Fibre losses and linear effects}

In modern communication applications, optical fibres impose as the best data transfer medium. The qualities of an optical fibre supersede those of electrical low- and high-frequency conductors. Firstly, optical fibres employ a much higher frequency range, enabling broad-spectra data transfer. The medium is, thus, utilized to a much greater extent than, for example, a coaxial cable operating in ultra-high frequency range under 1 GHz. Also, optical fibres have proven to be much more durable in terms of both mechanical endurance and immunity to electromagnetic interference. In communication and sensing applications, optical fibres represent a reliable, resilient medium. \\

In normal conditions, optical power in the fibre attenuates as governed by Beer's law. Optical power at any point in the fibre can be expressed by
\begin{equation}
P(z_0) = P_\textrm{in} \exp\left(-\alpha z_0\right) \textrm{,}
\end{equation}
where $\alpha$ is the attenuation coefficient in units of Np/km. For convenience, $\alpha$ is usually expressed in units of dB/km. Losses in a fibre depend on the optical wavelength. In communication systems, three \emph{optical communication windows} were defined as wavelengths where silica fibre exhibits minimum losses. The first around 850 nm (not as used in modern systems), the second around 1310 nm and the third around 1550 nm. These windows arise as a result of wavelength dependence of fibre material absorption, which represents the absorption by fused silica and due to the presence of impurities. The other fundamental loss mechanism is Rayleigh scattering. It arises from microscopic density fluctuations of silica that lead to refractive index fluctuations on a scale smaller than the optical wavelength. The contribution of Rayleigh scattering to the overall fibre loss can be expressed as
\begin{equation}
\alpha_\textrm{Ray} = C / \lambda^4 \textrm{.}
\end{equation}
Rayleigh scattering is, thus, largely dependent on wavelength. \\


\section{Non-linear optical effects}

\subsection{Phase-modulating effects}

Due to its imperfectly cylindrical shape, every fibre exhibits \textit{birefringence}. Two orthogonally polarized modes no longer propagate along the fibre independently, but instead exchange their powers, as presented in Figure \ref{fig:birefringence}.
%\label{fig:birefringence}
This is due to different refractive indices $\bar{n_x}$ and $\bar{n_y}$. They are commonly related by a quantity called \textit{birefringence}
\begin{equation}
B_m = \left| \bar{n_x} - \bar{n_y} \right| \textrm{.}
\end{equation}
The \textit{beat length} is obtained as
\begin{equation}
L_B = \lambda / B_m \textrm{.}
\end{equation}

Fundamental non-linear effects are self-phase modulation (abbr. SPM) and cross-phase modulation (abbr. XPM). The former is the result of a power-dependant propagation constant
\begin{equation}
\beta(P) = \beta + \frac{k_0 \bar{n_2} P}{A_\textrm{eff}} = \beta + \gamma P \textrm{.}
\end{equation}
The phase shift for SPM can be obtained as
\begin{equation}
\phi_\textrm{SPM} = \gamma P_\textrm{in} L_\textrm{eff} \textrm{,}
\end{equation}
where $L_\textrm{eff}$ is the effective interaction length, found as
\begin{equation}
L_\textrm{eff} = \frac{1 - \exp\left(-\alpha \, L\right)}{\alpha} \textrm{.}
\end{equation}
Due to birefringence, two orthogonal components will interact, causing XPM with phase shifts in each component
\begin{eqnarray}
\phi_x = \gamma \left(P_x + B P_y\right) L_\textrm{eff} & \phi_y = \gamma \left( P_y + B P_x \right) L_\textrm{eff} \textrm{.}
\end{eqnarray}
The total phase shift from SPM and XPM is given as
\begin{equation}
\varDelta \phi_\textrm{NL} = \gamma L_\textrm{eff} \left(1 - B\right) \left(P_x - P_y\right) \textrm{.}
\end{equation}

For long fibres, effective interaction length can be approximated as $L_\textrm{eff} = 1/\alpha$. For communication systems, the maximum tolerable phase shift is taken as $\phi_\textrm{NL} < 0.1$. Therefore, communication systems use power levels that are not to exceed
\begin{equation}
P_\textrm{in} < 0.1 \frac{\alpha}{\gamma N_A} \textrm{,}
\end{equation}
where $N_A$ is the number of amplifiers used. Therefore, SPM alone is a major limiting factor in long-haul optical systems.


\subsection{Brillouin scattering}

In distributed temperature sensing (abbr. DTS) systems, either of the two non-linear effects can be effectively used. The first is the Brillouin effect. This effect is a result of the electrostriction phenomenon, i.e. the compression of the material in the presence of an electromagnetic field. The optical pulse functions as a pump, exciting molecules in the fibre from the ground state into a higher, virtual state. The decay of these states will produce a Stokes component -- with the wavelength being longer that that of the incident optical wave -- or an anti-Stokes component, with the wavelength being shorter than that of the incident optical wave. An illustration of this shift is given in Figure \ref{fig:freq_shift}. 
%\label{fig:freq_shift}
This shift will create phonons, i.e. an acoustic wave at frequency
\begin{equation}
\Omega = \left|k_A\right| \, v_A = 2 v_A \, \left|k_p\right| \, \sin\left(\theta / 2 \right) \textrm{,}
\end{equation}
where wave vectors $k_A = k_P - k_S$ correspond to acoustic, pump and Stokes waves, $v_A$ the acoustic velocity, and the angle $\theta$ represents the angle between the pump and the scattered waves. Note that, due to the $\sin\left(\theta / 2\right)$ expression, the produced wave will not propagate in forward direction, but will instead be maximum in backward direction. The Brillouin shift is given as
\begin{equation}
\nu_B = \pm \frac{\Omega_B}{2 \pi} = \pm \frac{2 n_g v_A}{\lambda_P} \textrm{.}
\end{equation}
For silica fibre, and a typical pump wavelength of 1.55 \textmu m, the frequency is about 
\begin{equation}
\nu_B = \SI{11.1}{\giga \hertz} \textrm{.}
\end{equation}

The value of the frequency shift is temperature- and strain-dependent. Typically, the Brillouin frequency shift is tied to these physical constants at rates about $C_{\nu\theta} = \SI{1.10}{\mega \hertz \, \kelvin^{-1}}$ and $C_{\nu\epsilon} = \SI{48}{\kilo \hertz \, \epsilon^{-1}}$. The width of Stokes and anti-Stokes lines is typically 30--40 MHz. Monitoring the central frequencies, one can obtain the information about temperature and strain along the fibre. \\

The described method will be efficient in the spontaneous Brillouin scattering regime. In this regime, the scattered wave is generated spontaneously, as a result of fibre molecules being already thermally excited. Provided the incident, i.e. pump, optical power is large enough, the produced wave will beat with the pump wave, increasing the amplitude of the scattered wave. Such a phenomenon is then referred to as the stimulated Brillouin scattering (abbr. SBS). In order for the simulated scattering to take effect, the pump pulse power must be over the threshold level. This level is dependent on the Brillouin gain $g_B$, and the effective cross-section area of the fibre in question 
\begin{equation}
A_\textrm{eff} = \pi \, w^2 \textrm{,}
\end{equation}
with $w$ being the spot size. The threshold is given as
\begin{equation}
P_\textrm{th} \approx 21 \cdot \frac{A_\textrm{eff}}{g_B \, L_\textrm{eff}} \textrm{,}
\end{equation}
where $L_\textrm{eff}$ is the same as in \ref{eq:leff}. Some methods for measuring temperature or strain by using stimulated Brillouin scattering , employing an excitation laser at each end of the measurement fibre, exist %\cite Rogers 1999
. This, however, requires excellent source coherence, and it is also impossible to measure temperature and strain simultaneously. Such a measurement fibre could be hardly used in communication purposes in parallel. The stimulated Brillouin scattering method, also dubbed \textit{loss Brillouin OTDR}, can nevertheless prove advantageous in long-haul measurements.

\subsection{Raman scattering}
The other effect used in distributed temperature sensing systems is the Raman effect. This effect relies on molecular vibrations and rotations modulating the incident optical light. The same phenomemon of photon absorption and reemission occurs as in Figure \ref{fig:freq_shift}, only this time no acoustic phonons are produced, as opposed to Brillouin scattering. Therefore, Raman scattering is an isotropic process, and the scattered signals propagate in all directions. In this effect, the vibrational energy levels of the fibre structure dictate the value of Raman shift
\begin{equation}
\Omega_R = \omega_P - \omega_S \textrm{.}
\end{equation}
As in Brillioun effect, Stokes and anti-Stokes shifts are defined, with the photons produced by the former having a wavelength at frequency $\omega_S$ longer than the incident pump wavelength at frequency $\omega_P$, and those of the latter a shorter wavelength corresponding to frequency $\omega_{AS}$. The total optical power produced by the Stokes shift can be obtained from 
\begin{equation}
dP_S = P_0 \, \rho_S \, \Gamma_S \, dz \textrm{,}
\end{equation}
where $P_0$ is the incident optical pulse power, and $\Gamma_S$ the coefficient of scattering capture, dependent on the fibre's geometry and Stokes wavelength. The Bose-Einstein distribution $\rho_S$ is temperature dependent, and can be expressed as
\begin{equation} \label{eq:be-s}
\rho_S = \frac{1}{1- \exp\left( - \frac{\varDelta E}{k T} \right)} \textrm{.}
\end{equation}
The energy shift $\varDelta E$ can be expressed as
\begin{equation}
\varDelta E = \hbar \left( \omega_P - \omega_S \right) \textrm{.}
\end{equation}
A similar differential power expression can be expressed for the anti-Stokes shift, with a different term for the Bose-Einstein distribution
\begin{equation} \label{eq:be-as}
\rho_{AS} = \frac{\exp\left( - \frac{\varDelta E}{k T} \right)}{1 - \exp\left( - \frac{\varDelta E}{k T} \right)} \textrm{.}
\end{equation}
Here, $\varDelta E$ is defined as
\begin{equation}
\varDelta E = \hbar \left( \omega_P - \omega_AS \right) \textrm{.}
\end{equation}
The different distribution terms arise from the fact that the anti-Stokes radiation produces photons of larger energy than the incident pump photon. The energy difference must be provided by the excited molecule itself, and therefore the term must account for the number of molecules available in the proper vibrational state. It is now clear that both shifted components have a temperature dependence, but also that the dependence is much larger for the anti-Stokes component.

% Druge frekvencije od bs
% rs ide u svim smjerovima, bs samo u back


% Mjerenje temperature
% Ramanova metoda
% Objasniti fenomen


% Spontaneous and stimulated


% Sirokopojasnost Ramana

% --------------------------------------

\setcounter{stranica}{\thepage}
\addtocounter{stranica}{1}

\end{document}