\documentclass{standalone}
\usepackage{diplomski}

\begin{document}

\chapter{Laboratory setup}
\pagenumbering{arabic}
\setcounter{page}\thestranica

% --------------------------------------

In order to use Raman scattering for temperature measurement, a laser source must be capable of producing sufficiently short optical impulses, so as to meet spatial resolution requirements, as described by equation \ref{eq:otrd_spatial_resolution}. In this laboratory setup, an Optilab NPL-1550-37-R nanosecond pulse laser was used as the optical source. The laser is capable of producing pulses down to 2 ns in width, with a repetition rate of 100 Hz to 1 MHz. The energy of a single pulse can be up to 100 \textmu J. High optical power is achieved using a dual-stage amplifier, capable of producing a continuous optical power of 37 dBm, while operating in the 1543 -- 1570 nm region. High-power laser output is factory-couple into a single-mode fibre pigtail. Also, a 1\% tap FC output is available, which will be used in this setup for triggering the DAQ device. This signal will also be used to measure the peak optical power. \\

The measurement setup is displayed in Figure \ref{fig:lab_setup}.
%%\label{fig:lab_setup}
The laser's high-power output is protected from backscattered signals by an isolator placed immediately after the output pigtail. Although the majority of amplified spontaneous emission (abbr. ASE) from laser's integrated amplifiers is already filtered \cite{datasheet:laser}, an additional filter is placed after the isolator, to ensure the ASE is well attenuated. The selected filter is a dense wavelength-division multiplex (abbr. DWDM) filter, centred at \cite{datasheet:dwdm_filter}
\begin{equation}
\label{eq:dwdm_filter_centre}
\lambda_0 = \SI{1550}{\nano \meter}
\end{equation} and a pass bandwidth 
\begin{equation}
\label{eq:dwdm_filter_bw}
\varDelta f = \SI{200}{\giga \hertz} \textrm{.}
\end{equation}
Although the filter is specified to be able to withstand a continuous optical power of 500 mW, in pulse regime the expected tolerance to peak power is much higher. Connecting the rest of the measurement setup to the filter's \textit{pass} output, the initial ASE is now ensured to be negligible. \\

The circulator placed after the DWDM filter is an important component in the measurement setup. In the first setup, single-mode fibres were used as temperature sensing elements. Fibres coupled onto the circulator need to match the structure of the measurement fibre, so as to ensure good coupling and minimum losses of the excitation and reflected signals. The circulator itself is a low insertion loss-device and provides high isolation between ports \cite{datasheet:smf_circualtor}. The circulator has single-mode pigtails. Its port 1 was coupled to the high-power output from the DWDM filter, port 2 to the measurement fibre, and port 3 to the \textit{monitoring branch} of the system. The connection diagram is presented in Figure \ref{fig:circulator_connection}.
%\label{fig:circulator_connection}
% \\

Measurement fibres can be single rolls of fibre, or consist of a cascade of a number of fibre rolls. In the single-mode setup, regular SM fibres, compliant to the G.652 standard, were interchanged with highly non-linear fibres, i.e. fibres with a high $\left|\beta_3\right|$ value. The observed measurement-fibre system had a total length of about 5 km. Parts of this fibre system can be inserted into a cooling or heating chamber in order to control the system temperature. The far end of the measurement fibre has to be terminated to reduce reflection of the incident high optical power off the fibre end. Such a reflection could induce SRS at incident optical powers that are normally expected to be sufficiently low to keep the system in the spontaneous Raman-scattering regime. To achieve this, the one option is to use proper terminators. However, reflections can occur on unsuitable optical connectors, such as FC, that inherently exhibit high reflections. The other option is to couple the optical power into the environment, and ensuring power containment by external means. \\

The latter solution was used in the second scenario, where multi-mode fibres replaced single-mode fibres as sensing elements. Between the DWDM filter and the circulator, full optical power from the single-mode-coupled laser was then coupled into multi-mode fibres. A single-mode fibre-based circulator was replaced by a multi-mode version. The new circulator has a multi-mode fibre factory-spliced onto its ports. It also exhibits low insertion losses, and high isolation \cite{datasheet:mmf_circulator}. The high-power input is coupled into circulator port 1, and the fibre-under-test is directly coupled onto port 2. Port 3 was also coupled to the \textit{monitoring branch}. In a multi-mode scenario, the measurement fibre was expected to be able to withstand higher incident powers before exhibiting stimulated non-linear effects. Therefore, a standard G.651-compliant fibre was used for this purpose. A number of fibre rolls were spliced together to increase the total system length. Also, a short patch cable, 50 m in length, was inserted between two longer rolls. This simplified temperature control on a known point in the system. The patch was placed in a pot, in which heating and cooling of the fibre was performed by controlling the temperature of the water surrounding it. The isolation of the fibre patch was left unstripped to pertain its thermal and mechanical endurance. \\

The monitoring branch is connected to the circulator so that the scattered signal is coupled into it. Three backscattered components need to be demultiplexed to monitor every component separately. A Raman Wavelength Division Multiplexer, the so-called \textit{Raman cube}, was used to achieve this. The two versions, one for the single-mode scenario, and the other for the multi-mode scenario, have one common port and three other ports centred at different wavelengths. One port is used to monitor Rayleigh scattering, one to montior Stokes, and the other to monitor anti-Stokes components. The central-wavelengths for the three ports are
\begin{equation}
\lambda_\textrm{Ray} = \SI{1551}{\nano \meter}
\end{equation}
\begin{equation}
\lambda_\textrm{S} = \SI{1650}{\nano \meter}
\end{equation}
\begin{equation}
\lambda_\textrm{AS} = \SI{1450}{\nano \meter} \textrm{,}
\end{equation}
respectively. Some papers report the possibility of measurements of Raman-scattered components being biased due to the leakage of Rayleigh-scattered components through the Raman cube \cite{rayleigh_leakage}. The reports, however, used a filter whose isolation of Rayleigh component from Stokes and anti-Stokes was less than 30 dB. The Raman cube used in this thesis has the isolation of these critical couplings larger than 60 dB.

The Raman cube enables wavelength-demultiplexed scattered signals to be coupled into photodetectors. Three different InGaAs-based detectors by Thorlabs were tested here. Their optical and electrical characteristics are given in Table \ref{table:photodetectors}.
\begin{table}[h]
	\centering
	\caption{Optical and electrical characteristics of photodetectors}
	\label{table:photodetectors}
	\hspace*{-2em}
	\begin{tabular}{c|c|C{2cm}|C{2cm}|C{2cm}|C{2cm}|C{2cm}}
		\textbf{Model}	& \textbf{Type}	& \textbf{R [A/W] @ 1550$\,$nm}	& \textbf{Adjustable gain$\,$setting [dB]}	& \textbf{Transimp. gain [$\bm{\times 10^4 \textrm{V/A}}$]}	& \textbf{Bandwidth [MHz]}	& \textbf{NEP [$\bm{\textrm{pW}/\sqrt{\textrm{Hz}}}$]} \\ \hline \hline 
		PDA10CF-EC	& PIN	& 1.02	& --	& 1	& 150	& 12.0 \\ \hline
		\multirow{3}{*}{PDA10CS-EC}	& \multirow{3}{*}{PIN}	& \multirow{3}{*}{1.05}	& 20	& 1.5	& 1.9	& 3.0 \\ 
		& & & 30	& 4.75	& 0.775	& 1.25 \\
		& & & 40	& 15.1	& 0.320	& 1.40 \\ \hline
		APD130C	& APD	& 9	& --	& 10	& 50	& 0.46 \\
	\end{tabular}
\end{table}
For the PIN diode with switchable gain, the gain-bandwidth product is
\begin{equation}
\textrm{GBP} = \SI{600}{\mega \hertz} \textrm{.}
\end{equation}
Although these detectors are able to provide large gains, which are needed to monitor signals as weak as Stokes and anti-Stokes backscatterings, integrated amplifiers will not be able to provide sufficient bandwidth to faithfully detect the waveform of these signals. For the APD diode, the multiplication factor is
\begin{equation}
\textrm{M} = 10 \textrm{.}
\end{equation}
Although APD diodes normally exhibit larger noise in comparison to PIN diodes, this particular model provides very low noise powers, as well as inherent high gain and very large bandwidth. Therefore, in the final stage of this project, APD diode was installed to detect the anti-Stokes signal, as it has proved critical due to its low amplitude and high temperature-dependence. Additional optical components were secured to ensure proper coupling of the entire anti-Stokes signal power to the APD photodetector. These include an extra collimator, focusing lens, and precision mechanical focusing components. The Rayleigh scattering component was not monitored in this setup. \\

Photodetectors have a BNC connector at their output. This enables them to operate under a matched 50 \textOmega or a high-impedance load. The transimpedance gain presented in table \ref{table:photodetectors} is the gain achieved with the high-impedance load on the photodetector's output, as was the case in this laboratory setup. All the photodetectors' output were monitored by a DAQ device PXIe-5160 from National Instruments. The device is a 10-bit analog-to-digital converter, its highest input frequency being
\begin{equation}
f_\textrm{max} = \SI{500}{\mega \hertz} \textrm{,}
\end{equation}
the configured input impedance
\begin{equation}
Z_\textrm{in} = \SI{1}{\mega \ohm} \| \SI{15}{\pico \farad} \textrm{,}
\end{equation}
and coupling set to AC. The sampling rate is
\begin{equation}
f_\textrm{S} = \SI{1.25}{\giga S / \second} \textrm{,}
\end{equation}
which ensures a faithful representation of the signal detected by the photodetector. The PXIe-5160 has an extension card form-factor, and is hosted by a National Instruments PXIe-1082 chassis. The system controller is also present in one of the chassis's slots. Two channels, CH2 and CH3, of the DAQ device are used to monitor the Stokes and anti-Stokes signal, respectively. To the channel CH4, the output signal from an Analog Devices TMP36 calibrated temperature sensor is connected. This device is used to calibrate the temperature offset of the system, by determining the temperature at the beginning of the measurement fibre. The TMP36 is soldered onto a coaxial, and a USB cable in parallel. This enables power supply from the system controller's USB hub. The current temperature of the TMP36 can be calculated as \cite{datasheet:tmp36}
\begin{equation}
T \, \textrm{[\textdegree C]} = \left( V \, \textrm{[mV]} - V_\textrm{n} \right) \cdot 0.1 \, \textrm{[\textdegree C / mV]} + 25 \, \textrm{\textdegree C}
\end{equation}
\begin{equation}
V_\textrm{n} = \SI{750}{\milli \volt} \textrm{.}
\end{equation}

To the CH1 channel of the DAQ card, a simple PIN photodiode, model PDA10CF-EC from Thorlabs, is connected. Its optical input is connected to the 1\% tap output from the laser. The signal is attenuated before reaching the photodetector by an attenuator that operates in the range
\begin{equation}
L_\textrm{att} \in [\SI{25}{\decibel}, \SI{30}{\decibel}] \textrm{.}
\end{equation}
The signal on channel CH1 is used for triggering the data acquisition at the same time as the optical pulse is sent along the measuring fibre. Also, knowing the opto-electrical characteristics of the photodetector, one can use the measured peak value of the optical pulse to determine the peak optical power in the fibre. By monitoring the waveform acquired by the DAQ card, one can also observe the shape of the optical pulse propagating along the fibre. \\

The described laboratory setup enables simple replacement of photodiodes, testing of different fibre types and combinations, and reconfiguration of the digital processing algorithm. Optical parts are connected by SC or FC connectors. These need to be thoroughly cleaned to ensure best performance. All the relevant optical parameters in the system can be monitored. 

% --------------------------------------

\setcounter{stranica}{\thepage}
\addtocounter{stranica}{1}

\end{document}