\documentclass{standalone}
\usepackage{diplomski}

\begin{document}

\chapter{Conclusion}
\pagenumbering{arabic}
\setcounter{page}\thestranica

% --------------------------------------

In this thesis, an implemented fibre-optical sensing system was described. Distributed temperature sensing was achieved by utilizing the Raman scattering effect, observing it by a modified optical time-domain reflectometer. In the introduction, the possibility of using optical fibres in sensing applications is recognized and identified as a technologically and practically superior method in a variety of scenarios, compared to classic electronic sensors. Chapter \ref{ch:fibres} describes the theory behind optical fibres, and the basis for their use in communication systems. The most significant linear and non-linear optical effects in fibres are discussed, including relevant scattering effects that enable the use of reflectometric methods. In chapter \ref{ch:sensing_method}, these effects are put into use for measuring fibre characteristics, and finally, obtaining the temperature profile. Operating principles of two types of reflectometers, using time-domain and frequency-domain method, were described. The implemented laboratory setup is described in detail in chapter \ref{ch:setup}. The data acquisition and processing algorithm was developed ground-up. The development process included testing various noise cancellation methods, which are described in chapter \ref{ch:data_processing}. \\

Several setups, including single-mode or multi-mode-based systems were used to perform system calibration and temperature measurements. The second setup, with multi-mode fibres as sensing elements, proved superior to the single-mode setup, in terms of achieved temperature accuracy and tolerance to calibration errors. Measurement results are presented and discussed in chapter \ref{ch:results}. Maximum laser optical powers were measured to keep the system in the spontaneous Raman scattering regime. Other system configuration parameters, such as optical pulse widths, were also evaluated and their influence on system performance qualitatively measured. An optimal configuration and system setup was found, and final temperature measurements were presented. In chapter \ref{ch:discussion}, current system's disadvantages were identified and a number of enhancements were proposed. Some of the approaches were conceptually evaluated, but left unimplemented in the current system version. \\

The implemented system proved high temperature and spatial accuracies can be gained with proper selection of measurement fibres. It is shown that photodetectors play a key role in determining system performance, as the observed signals are very weak and contain high-frequency components. Although fast data acquisition systems were used to monitor scattered signals, the data processing platform had shown some performance issues. A faster DAQ-to-system-controller interface should be implemented in further development. \\

Distributed sensing systems provide a much better insight into environmental conditions on a given line. Fibre sensing technology enables the use of one optical fibre for multiple sensing and communication purposes in parallel. Investing into such systems represents a dual benefit for the investor, especially within large infrastructure projects. Furthermore, sturdiness of optical fibres, as well as their inherent immunity to electro-magnetic and environmental interferences make fibre-optic-based sensing technologies highly reliable. Fast development of electronics and optical technology raises component availability, and distributed sensing systems are expected to be more present in upcoming years.


% --------------------------------------

\setcounter{stranica}{\thepage}
\addtocounter{stranica}{1}

\end{document}