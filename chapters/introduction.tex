\documentclass{standalone}
\usepackage{diplomski}

\begin{document}

\chapter{Introduction}
\pagenumbering{arabic}
\setcounter{page}\thestranica

% --------------------------------------

In modern communication systems, optical fibres are the most widespread medium for transmitting optical signals. They feature minimal losses, as well as high resistance to environmental conditions surrounding them. Modern communication systems enable up to 10 Tbit/s links over a single fibre thread. These features, along with a very fast development of the technology during the last 20 years, have placed fibre communication systems on virtually all backbones of modern networks. Commercially available fibre-to-the-home, curb, or building infrastructure solutions bring high speed network access to all users. New generations of radio-access networks enable high speed wireless access, but are reliant on optical network to ensure high-data-rate access for its users. Constant advancements in access networks, and the growing demand to connect a large number of devices transferring a vast amount of data, are pushing the optical technology towards greater speeds and reliability, as well as its availability throughout the world. \\

When used in optical communication systems, fibres are ensured to operate in a regime that will minimize unwanted effects that originate from fibre structure. Firstly, optical powers are kept sufficiently low, so as not to induce unwanted non-linear effects. Furthermore, high-data-rate communications employ a series of data recovery algorithms, in an effort to mitigate the unavoidable effects that may deteriorate the signal. The final goal is to keep the signal-to-noise ratio (abbr. SNR) as large as possible, while avoiding dispersion. In the eye-diagram representation of transmission quality, the desired effect is to keep the eye \textit{widely open}. Many of these and other opto-mechanical effects are the result of physical processes in fibres. Fibre-sensing technologies utilize these effects, and try to measure the relevant physical quantities that caused these effects. A result of such measurements can be the characterization of either the environment or the measurement fibre itself. \\

The observed physical quantity can be temperature, strain, vibration, etc. Such sensors can be applied in a variety of scenarios, some of which were hardly manageable with classic electrical sensors. Examples include monitoring the structural integrity of large objects, such as bridges, buildings, etc. Temperature sensors can be used to monitor the environment inside large objects such as tunnels, ships, or industrial facilities. Optical sensors can be implemented as distributed sensors that display the physical quantity along the entire length of the measurement medium, as opposed to discrete sensors, whose measuring capability is limited to one point in space. A distributed temperature sensor can, thus, not only detect a fire hazard in a tunnel, but also pinpoint the dangerous area with a resolution of a couple of metres. Depending on the sensing method, it may even be sufficient to approach the measurement fibre only from the one side, keeping the far end terminated or properly adapted for other parallel uses. The monitoring station can, in fact, be tens of kilometres away from the area of interest, and the whole system can function with no additional power supply at the measurement site. The sensing fibre could also be used for communication in parallel, provided the isolation between the two co-hosted systems is well-implemented. \\

This paper overviews distributed optical fibre sensors (abbr. DOFS). The theoretical background to optical fibres and their sensing capabilities is provided. A distributed temperature sensor based on Raman scattering is implemented and discussed. Measurements presented in this thesis are aimed towards performance comparison of different system configurations.


% --------------------------------------

\setcounter{stranica}{\thepage}
\addtocounter{stranica}{1}

\end{document}